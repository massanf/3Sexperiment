\documentclass{ltjsarticle}
\usepackage{luatexja}
%Packages -----------------------
\usepackage{titlesec}
\usepackage{listings}
\usepackage[euler]{textgreek}
\usepackage{enumitem}
\usepackage[margin=30mm]{geometry}
\usepackage{comment}
\usepackage{hyperref}
\usepackage{float}
\usepackage{textcomp}
\usepackage{xparse}
%Circuit ------------------------
\usepackage{amsmath,amssymb}
\usepackage{graphicx}
\usepackage{siunitx}
\usepackage{tikz}
\usepackage[siunitx, RPvoltages]{circuitikz}
%Circuit ------------------------
%Command ---–--------------------
\renewcommand{\figurename}{図}
\renewcommand{\baselinestretch}{1.1}
\lstset{
    numbers=left,
    basicstyle={\ttfamily},
    identifierstyle={\small},
    commentstyle={\smallitshape},
    keywordstyle={\small\bfseries},
    ndkeywordstyle={\small},
    stringstyle={\small\ttfamily},
    frame={tb},
    breaklines=true,
    columns=[l]{fullflexible},
    xrightmargin=0\zw,
    xleftmargin=3\zw,
    numberstyle={\scriptsize},
    stepnumber=1,
    %numbersep=1,
    lineskip=-0.5ex,
    keywordstyle=\color[HTML]{e10021},
    commentstyle=\color{gray},
    emph=CascadeObjectDetector,
    emphstyle=\color{blue}
}
%Command ------------------------
%Title   ------------------------
\begin{titlepage}
\title{実験レポート E2}
\author{東京大学工学部電気電子工学科 03210517 \\ 藤田 誠之 \\共同実験者  河野亮介・梶谷優貴・パリナヨックサタユ}
\date{July 6, 2021}
\end{titlepage}
%Title   ------------------------

\begin{document}

\maketitle

\section{実習課題}
\subsection{等価回路作成}
発電機が接続されていない無負荷実験1のデータを用いて、横軸を電圧、縦軸を有効電力としてプロットすると以下のようになる。また、最小二乗法で2次で近似した線も表示する。
\begin{figure}[H]
    \begin{center}
        \includegraphics[width=11cm]{figures/無負荷試験1.png}
        \caption{無負荷試験}
    \end{center}
\end{figure}
近似の式は以下のようになった。
\[
    P = G_0 V^2_1 + P_m \\
      = 0.00466 V^2_1 - 0.162 V  + 61.1
\]
よって、$G_0 = 0.00466$と求まる。また、
\[
    B_0 = \sqrt{3\left(\frac{I0}{V1}\right)^2-G^2_0} = 0.05088
\]
さらに、拘束試験から$r_1+r'_2$、$x_1+x'_2$を求めることができる。
\[
    r_1+r'_2 = \frac{P}{3I^2_1} = 1.37
\]
\[
    x_1+x'_2 = \sqrt{\frac{V^2_1}{3I^2_1}-(r_1+r'_2)^2} = 1.93
\]
速度特性は以下のようになった。

 

\section{検討事項}
\subsection{}
負荷試験の結果と簡易等価回路から求めた結果とを比較し、その差の原因について考察せよ。\\~\\

無負荷試験、拘束試験から、簡易等価回路は以下のように求まった。
\begin{figure}[H]
    \begin{center}
        \begin{circuitikz}[american currents,american resistors]
            \ctikzset{american inductors,resistor=european}
            \draw (0,0)
            to[short] ++(2,0)
            to[short] ++(0,1)
            to[short] ++(-0.5,0)
            to[R=$G_0$] ++(0, 2)
            to[short] ++ (0.5,0)
            to[short] ++ (0,1)
            to[short] ++ (-2,0);
            \draw (2,3)
            to[short] ++(0.5,0)
            to[L=$B_0$] ++(0,-2)
            to[short] ++(-0.5,0);
            \draw (2,4)
            to[R,l_=$r_1+r'_2$] ++(2.5,0)
            to[L,l_=$x_1+x'_2$] ++(2.5,0);
            \draw(2,0)
            to[short] ++(5,0)
            to[vR,l_=$r'_2\frac{1-s}{s}$] ++(0,4);
        \end{circuitikz}
        \caption{L型簡易等価回路 回路図}
    \end{center}
\end{figure}

\[
G_0 = \SI{4.66}{m\Omega}, B_0 = \SI{0.509}{H} r_1=r'_2 = \SI{0.817}{\Omega}, x_1+x'_2 = \SI{2.125}{H}
\]

これを用いて、以下のようにトルクを計算する。
\[
T(s) = \frac{1}{2\pi\frac{f}{p}}\frac{3V^2_1}{\left(r_1+\frac{r'_2}{s}\right)^2+(x_1+x'_2)^2}\frac{r'_2}{s}
\]
\[
f=\SI{50}{Hz}, p=2, V_1 = \SI{200}{V}
\]
これが、簡易等価回路から求められる速度特性である。\\
また、実測値をプロットする。滑りは以下の式で求めることができる。
\[
s = \frac{n_s-n}{n_s}
\]
\[
n_s = \SI{1500}{rpm}    
\]
トルクは、DCの値から以下のように求められる。
\[
T = \frac{P_2}{2\pi\frac{n}{60}} = \frac{30V_2I_2}{\pi n}
\]
また、力を用いると以下のようにも求められる。
\[
T = F \cdot L    
\]
\[
L = \SI{0.287}{m}
\]
これらをすべて一つのグラフにプロットすると、以下のようになる。
\begin{figure}[H]
    \begin{center}
        \includegraphics[width=12cm]{figures/2/results.png}
        \caption{速度特性}
    \end{center}
\end{figure}
プロットすると、実験値のトルクは等価回路で求められたトルクよりも小さくなっていることがわかる。等価回路は実物ではなくあくまでもモデルであるため、全くそのままの出力が出ることは期待できないとした上で考える。
拘束試験においては$I_1 = I'_1$と仮定しているため、
\[
    P = 3(r_1+r_2)I^2_1
\]
の$I_1$の値が実際より大きい。つまり$r_1$と$r_2$の値が小さく見積もられているため、計算したトルクは実際の値より大きくなるはずである。このことが図2に現れていると考えられる。

\subsection{}
誘導機の等価回路を図 2 に示したものではなく,図 9 に示した T 型等価回路としたときの諸量の速度特性をグラフに示し,図 2 に示した等価回路から導いた結果との差について考察せよ.

負荷実験では励磁電流$I_0$は電機子電流全体$I'_1$に対して十分小さいため、励磁電流の位置を移しても問題ない(というより、T型の方が実際の構造に近く、励磁電流が電気子電流全体に比べて十分に小さいためL型の等価回路を用いて計算しても計算しても大きな結果の差が出ないとしてL型が使われている)。

\begin{figure}[H]
    \begin{center}
        \begin{circuitikz}[american currents]
            \ctikzset{american inductors,resistor=european}
            \draw (0,0)
            to[short] (4,0)
            to[short] ++(0,1)
            to[short] ++(-0.5,0)
            to[R=$G_0$] ++(0, 2)
            to[short] ++ (0.5,0)
            to[short] ++ (0,1);
            \draw (0,4)
            to[R,l_=$r_1$] ++ (2,0)
            to[L,l_=$x_1$] ++ (2,0);
            \draw (4,3)
            to[short] ++(0.5,0)
            to[L=$B_0$] ++(0,-2)
            to[short] ++(-0.5,0);
            \draw (4,4)
            to[R,l_=$r'_2$] ++(2,0)
            to[L,l_=$x'_2$] ++(2,0);
            \draw(4,0)
            to[short] ++(4,0)
            to[vR,l_=$r'_2\frac{1-s}{s}$] ++ (0,4);
        \end{circuitikz}
        \caption{T型簡易等価回路 回路図}
    \end{center}
\end{figure}
L型と同じように計算すると、
\[
    I'_1 = \frac{V_2}{\sqrt{\left(\frac{r'_2}{s}\right)^2+x'^2_2}}
\]
また、
\[
    Y = G_0 + jB_0
\]
\[
    Z_1 = r_1+jx_1
\]    
\[
    Z_2 = \frac{1}{Y}
\]
\[
    Z_3 = r'_2 + jx'_2
\]
として、
\[
    V_2 = \frac{\frac{Z2 Z3}{Z2+Z3}}{Z1+\frac{Z2 Z3}{Z2+Z3}} V
\]
さらに
\[
    P_{out} = 3I'^2_1\frac{r'_2}{s}(1-s)
\]

\[
    T = \frac{1}{2\pi\frac{f}{p}}\frac{V^2_2}{\left(\frac{r'_2}{s}\right)^2+x'^2_2}\frac{r'_2}{s}
\]
このように計算すると以下のようになった。おそらく何かがおかしい。
\begin{figure}[H]
    \begin{center}
        \includegraphics[width=12cm]{figures/2/results_t.png}
        \caption{*****}
    \end{center}
\end{figure}

\subsection{}
トルクの差を考える。上で述べたように、トルクは電気量と力から求められる。
\[
    T = \frac{30V_2I_2}{\pi n} = F\cdot L
\]
\begin{figure}[H]
    \begin{center}
        \includegraphics[width=12cm]{figures/2/experiments.png}
        \caption{*****}
    \end{center}
\end{figure}

sが十分小さい範囲においては直線に近似できるため、$T = as + b$として傾きを最小2乗法で近似すると以下のようになった。

\begin{table}[H]
    \begin{center}
        \begin{tabular}{|c|c|c|c|} \hline            
            & a & b & 傾きの誤差率\\ \hline
            等価回路 & 228.9 & 0.0550 &  \\ \hline
            回転力から求めたトルク & 195.4 & -1.52 & 14.6\% \\ \hline
            電気量から求めたトルク & 258.4 & -2.30 & 12.9\% \\ \hline
        \end{tabular}
        \caption{*****}
    \end{center}
\end{table}

この結果から、電気量から計算されるトルクの方が誤差が大きいことがわかる。トルクという物理量を計測するにあたって出力された力を直接計測する方が、電気量から間接的に計算するよりも誤差が少ないということがわかる。

\subsection{}
誘導電動機には安定に運転可能な速度 (すべり) の範囲があるが,その理由を説明せよ.説明においては適当な負荷トルク特性を仮定せよ.\\

誘導電動機の速度特性に対して、適当な負荷トルクを仮定する。これを下図の点線で表す。

\begin{figure}[H]
    \begin{center}
        \includegraphics[width=11cm]{figures/2/experiments_.png}
        \caption{*****}
    \end{center}
\end{figure}

安定に動作するためには、まず出力トルクと負荷トルクが一致しなければならないため、考えられるのは実線と点線の交点である。それぞれ点A,点Bとする。

点Aは不安定な安定である。速度が上昇したとすると負荷トルクが出力トルクより大きくなる。そのため、電動機はさらに加速を進めてしまう。また、速度が減少したとすると負荷トルクが出力トルクよりも小さくなり、電動機はさらに減速してしまう。

一方、点Bは安定な安定である。速度が上昇したとすると負荷トルクが出力トルクより小さくなり、減速する。また、速度が減少したとすると負荷トルクが出力トルクよりも大きくなり加速する。

そのため、以下のように最大トルクを与えるすべりよりもすべりが小さい範囲において安定な運転が可能である。



\section{参考文献}
『誘導機の等価回路』\url{http://energychord.com/children/energy/motor/ind/contents/ind_eq.html} 2021年7月7日アクセス\\
url{https://www.tetras.uitec.jeed.go.jp/files/data/200506/20050614/20050614.pdf}\\
url{https://e-sysnet.com/三相誘導電動機の特性/}
\end{document}