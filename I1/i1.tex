\documentclass{ltjsarticle}
\usepackage{luatexja}
%Packages -----------------------
\usepackage{listings}
\usepackage[euler]{textgreek}
\usepackage{enumitem}
\usepackage[margin=30mm]{geometry}
\usepackage{comment}
\usepackage{hyperref}
\usepackage{float}
\usepackage{textcomp}
\usepackage{xparse}
%Circuit ------------------------
\usepackage{amsmath,amssymb}
\usepackage{graphicx}
\usepackage{siunitx}
\usepackage{tikz}
\usepackage[siunitx, RPvoltages]{circuitikz}
%Circuit ------------------------
%Command ---–--------------------
\renewcommand{\figurename}{図}
\renewcommand{\baselinestretch}{1.1}
\lstset{
    numbers=left,
    basicstyle={\ttfamily},
    identifierstyle={\small},
    commentstyle={\smallitshape},
    keywordstyle={\small\bfseries},
    ndkeywordstyle={\small},
    stringstyle={\small\ttfamily},
    frame={tb},
    breaklines=true,
    columns=[l]{fullflexible},
    xrightmargin=0\zw,
    xleftmargin=3\zw,
    numberstyle={\scriptsize},
    stepnumber=1,
    %numbersep=1,
    lineskip=-0.5ex,
    keywordstyle=\color[HTML]{e10021},
    commentstyle=\color{gray},
    emph=CascadeObjectDetector,
    emphstyle=\color{blue}
}
%Command ------------------------
%Title   ------------------------
\title{実験レポート I1}
\author{東京大学工学部電気電子工学科 03210517\ 藤田 誠之 }
\date{ May 20, 2021}
%Title   ------------------------
\begin{document}
\maketitle
\section{実験目的}
実験の目的
\section{1日目: 間違い探し}
\subsection{00}
stdio.hがインクルードされていないため正常に動作しない。一行目に以下のコードを追加すれば想定通りに動く。
\begin{lstlisting}[caption=p00.c 追加部分,language=C]
#include <stdio.h>
\end{lstlisting}
\subsection{01}
sinやcosといった関数が使われているにも関わらず、math.hがインクルードされていないためエラーとなる。以下のコードを追加すれば設計通りに動作する。
\begin{lstlisting}[caption=p01.c 追加部分language=C]
#include <math.h>
\end{lstlisting}
\subsection{02}
printfはstdio.h、atofはstdlib.hがそれぞれインクルードされていないと使うことができない。冒頭に以下のコードを追加すれば設計通りに動作する。
\begin{lstlisting}[caption=p02.c 追加部分,language=C]
#include <stdio.h>
#include <stdlib.h>
\end{lstlisting}
\subsection{03}
pがmallocされていないため設計通りに動作しない。

\begin{lstlisting}[caption=p02.c 追加部分,language=C]
vect3 * mk_point(double x, double y, double z)
{
  vect3 * p;
  //p = malloc(sizeof(vect3));
  p->x = x;
  p->y = y;
  p->z = z;
  return p;
}
\end{lstlisting}
\subsection{04}




\section{参考文献}
『title』\url{<url>} 2021年n月m日アクセス
\end{document}