\documentclass{ltjsarticle}
\usepackage{luatexja}
%Packages -----------------------
\usepackage{listings}
\usepackage[euler]{textgreek}
\usepackage{enumitem}
\usepackage[margin=30mm]{geometry}
\usepackage{comment}
\usepackage{hyperref}
\usepackage{float}
\usepackage{textcomp}
\usepackage{xparse}
%Circuit ------------------------
\usepackage{amsmath,amssymb}
\usepackage{graphicx}
\usepackage{siunitx}
\usepackage{tikz}
\usepackage[siunitx, RPvoltages]{circuitikz}
%Circuit ------------------------
%Command ---–--------------------
\renewcommand{\figurename}{図}
\renewcommand{\baselinestretch}{1.1}
\renewcommand\lstlistingname{ソースコード}
\renewcommand\lstlistlistingname{ソースコード}
\lstset{
    numbers=left,
    basicstyle={\ttfamily},
    identifierstyle={\small},
    commentstyle={\smallitshape},
    keywordstyle={\small\bfseries},
    ndkeywordstyle={\small},
    stringstyle={\small\ttfamily},
    frame={tb},
    breaklines=true,
    columns=[l]{fullflexible},
    xrightmargin=0\zw,
    xleftmargin=3\zw,
    numberstyle={\scriptsize},
    stepnumber=1,
    %numbersep=1,
    lineskip=-0.5ex,
    keywordstyle=\color[HTML]{e10021},
    commentstyle=\color{gray},
    emph=CascadeObjectDetector,
    emphstyle=\color{blue}
}
%Command ------------------------
%Title   ------------------------
\title{実験レポートA3}
\author{東京大学工学部電気電子工学科 03210517\ 藤田 誠之 }
\date{May 6, 2021}
%Title   ------------------------

\begin{document}

\maketitle
\section{方法}
Xilinx Vivadoを用いてHDLシミュレーションを実行した.
\section{考察課題}
\subsection{1日目}
\subsubsection{AND回路の設計}

入力は2つのwire, 出力は1つのwireである. 2つの入力の論理積を出力する。
\begin{lstlisting}[caption=AND回路デザイン,language=verilog]
`timescale 1ns / 1ps
module and_gate (
    input wire inA,
    input wire inB,
    output wire out
);
    assign out = inA & inB;
endmodule
\end{lstlisting}
2入力の全部で4通りがある。テストベンチで100nsごとに入力を変化させた。
\begin{lstlisting}[caption=AND回路テストベンチ,language=verilog]
    module testbench;
    // parameter
    parameter CYCLE = 1000; // clock cycle
    parameter HALF_CYCLE = 500; // half cycle
    parameter DLY = 500; // delay
    // wire/reg
    reg clk;
    reg inA, inB;
    wire out_and_gate;
    // DUT module
    and_gate and_gate_0 (
        .inA(inA),
        .inB(inB),
        .out(out_and_gate)
    );
    // clock generator
    always begin
        clk = 1'b1;
        #(HALF_CYCLE) clk = 1'b0;
        #(HALF_CYCLE);
    end
    // test scenario
    initial begin
        // initialize
        inA = 1'b0; inB = 1'b0; 
        // for and_gate
        inA = 1'b0; inB = 1'b0;
        #100 $display("inA=%b inB=%b out=%b", inA, inB, out_and_gate);
        inA = 1'b1; inB = 1'b0;
        #100 $display("inA=%b inB=%b out=%b", inA, inB, out_and_gate);  
        inA = 1'b0; inB = 1'b1;
        #100 $display("inA=%b inB=%b out=%b", inA, inB, out_and_gate);  
        inA = 1'b1; inB = 1'b1;
        #100 $display("inA=%b inB=%b out=%b", inA, inB, out_and_gate);

        repeat(10) @(posedge clk); // repeat 10 times
        $finish;
    end
endmodule
\end{lstlisting}

\begin{figure}[H]
    \begin{center}
        \includegraphics[width=14cm]{figures/and.png}
        \caption{AND回路の波形}
    \end{center}
\end{figure}
inA, inBの値が(0,0),(1,0),(0,1),(1,1)と変化するにつれ出力が変化していることがわかる。outの値が1となっているのはinAとinBが1のときのみであり、論理積が表現できていることがわかる。

\subsubsection{NAND回路の設計}
入力は2つのwire, 出力は1つのwireである. 2つの入力をA,Bとすると、$\overline{A\dot B}$を出力する。

\begin{lstlisting}[caption=NANDデザイン,language=verilog]
`timescale 1ns / 1ps
module nand_gate (
    input wire inA,
    input wire inB,
    output wire out
);
    assign out = ~(inA & inB);
endmodule
\end{lstlisting}
テストベンチはAND回路のときと同じものを用いた。

\begin{figure}[H]
    \begin{center}
        \includegraphics[width=14cm]{figures/nand.png}
        \caption{NAND回路の波形}
    \end{center}
\end{figure}
inA, inBの値を(0,0),(1,0),(0,1),(1,1)と変化するにつれ出力が変化していることがわかる。outの値は(0,0),(1,0),(0,1)のときに1、(1,1)のときに0となっているが、これは$\overline{A\dot B}$をこの回路が正しく表現できていることがわかる。

\subsubsection{XOR回路の設計}
入力は2つのwire, 出力は1つのwireである。2つの入力をA,Bとすると、$A \oplus B$を出力する。
\begin{lstlisting}[caption=XORデザイン, language=verilog]
`timescale 1ns / 1ps
module xor_gate (
    input wire inA,
    input wire inB,
    output wire out
);
    assign out = inA ^ inB;
endmodule
\end{lstlisting}
テストベンチはAND回路のときと同じものを用いた。
\begin{figure}[H]
    \begin{center}
        \includegraphics[width=14cm]{figures/xor.png}
        \caption{XOR回路の波形}
    \end{center}
\end{figure}
inA, inBの値を(0,0),(1,0),(0,1),(1,1)と変化するにつれ出力が変化していることがわかる。outの値はinA,inBが(1,0),(0,1)のときに1となっており、$A \oplus B$が表現できていることがわかる。
\subsubsection{フリップフロップ回路の設計}
\tikzset{sr-ff/.style={flipflop, flipflop def={
    t1=D,t3 = {\texttt{CLK}}, t6=Q, c3=1, td=rst}},
}
以下のフリップフロップ回路は、出力を反転して入力に戻す。
\begin{figure}[H]
    \begin{center}
        \begin{circuitikz}[american currents]
            \ctikzset{american inductors}
            \draw (0,0)
            node[sr-ff](FF){} (FF.up)
            (FF.pin 1) -- ++(-0.5, 0)
            to[short] ++(0, 2)
            to[short] ++(1,0) node[not port, anchor = out, rotate = 180](not1){}
            (FF.pin 6) -- ++(0.5, 0)
            to[short] ++(0, 2)
            to (not1.in) -- ++(0.5,0);
        \end{circuitikz}
        \caption{FF回路}
    \end{center}
\end{figure}
このFF回路を作成する。入力はclkとrstの2つのwireであり、出力はoutのreg一つである。clkの立ち上がりで出力値がそれまでの出力値の反転となる。ただし、clkの立ち上がり時にrstが1であればoutの値は0のままとなる。また、rstの立ち上がり時にも出力値が0となる。なお、regを用いてoutの値を保持しておくことで、Dの入力のwireが不要となる。
\begin{lstlisting}[caption=インバータ付きFF回路デザイン,language=verilog]
`timescale 1ns / 1ps
module flipflop (
    input wire clk,
    input wire rst,
    output reg q
);
initial begin
    q = 1;
end
always @(posedge clk or posedge rst)
    if (rst) begin
        q <= 1'b0;
    end else begin
        q <= ~q;
    end
endmodule
\end{lstlisting}
clkは500msごとに値が1と0で切り替わる。適当なタイミングでrstの値を1とし、rstが想定通りに機能していることを確認する。
\begin{lstlisting}[caption=インバータ付きFF回路テストベンチ,language=verilog]
    module testbench_flipflop_not;
    // parameter
    parameter CYCLE = 1000; // clock cycle
    parameter HALF_CYCLE = 500; // half cycle
    parameter DLY = 500; // delay
    // wire/reg
    reg clk;
    reg rst;
    wire out;
    // DUT module
    flipflop_not flipflop_not0(
        .clk(clk),
        .rst(rst),
        .q(out)
    );
    // clock generator
    always begin
        clk = 1'b1;
        #(HALF_CYCLE) clk = 1'b0;
        #(HALF_CYCLE);
    end
    // test scenario
    initial begin   
        rst = 1'b0;
        #2200 $display("rst=%b out=%b", rst, out);
        rst = 1'b1;
        #400 $display("rst=%b out=%b", rst, out);  
        rst = 1'b0;
        #600 $display("rst=%b out=%b", rst, out);
        rst = 1'b1;
        #400 $display("rst=%b out=%b", rst, out);
        rst = 1'b0;
        #1200 $display("rst=%b out=%b", rst, out);
        rst = 1'b1;
        #400 $display("rst=%b out=%b", rst, out);
        rst = 1'b0;
        #1200 $display("rst=%b out=%b", rst, out);
        repeat(10) @(posedge clk); // repeat 10 times
        $finish;
    end
endmodule
\end{lstlisting}
\begin{figure}[H]
    \begin{center}
        \includegraphics[width=14cm]{figures/flipflop_not.png}
        \caption{インバータ付きフリップフロップ回路の波形}
    \end{center}
\end{figure}
波形を観察すると、設計通り、clkの立ち上がりでoutの値が反転しており、またrstの立ち上がりでoutの値が0となっていることがわかる。


\subsubsection{ラッチ回路の設計}
\tikzset{sr-ff2/.style={flipflop, flipflop def={
    t1=D,t3 = {\texttt{CLK}}, t6=Q, c3=1}},
}
以下のようなラッチ回路を作成する。
\begin{figure}[H]
    \begin{center}
        \begin{circuitikz}[american currents]
            \ctikzset{american inductors}
            \draw (0,0)
            node[sr-ff2](FF){Latch} (FF.up);
        \end{circuitikz}
        \caption{ラッチ回路}
    \end{center}
\end{figure}
入力はclkとDのwireの2つ、出力は1つのregである。クロックが1であれば出力の値がDの値となり、クロックが0になってもその直前の出力値を保持する。
\begin{lstlisting}[caption=ラッチ回路デザイン,language=verilog]
`timescale 1ns / 1ps
module latch (
    input wire clk,
    input wire D,
    output reg Q
);
    always @* begin
        if (clk==1'b1)
        Q = D;
    end
endmodule
\end{lstlisting}
clkは250msごとに切り替わる。Dを適当なタイミングで切り替え、out値の挙動を観察する。
\begin{lstlisting}[caption=*****,language=verilog]
module testbench_flipflop_not;
    // parameter
    parameter CYCLE = 1000; // clock cycle
    parameter HALF_CYCLE = 500; // half cycle
    parameter DLY = 500; // delay
    // wire/reg
    reg clk;
    reg rst;
    wire out;
    // DUT module
    flipflop_not flipflop_not0(
        .clk(clk),
        .rst(rst),
        .q(out)
    );
    // clock generator
    always begin
        clk = 1'b1;
        #(HALF_CYCLE) clk = 1'b0;
        #(HALF_CYCLE);
    end
    // test scenario
    initial begin
        rst = 1'b0;
        #2200 $display("rst=%b out=%b", rst, out);
        rst = 1'b1;
        #400 $display("rst=%b out=%b", rst, out);  
        rst = 1'b0;
        #600 $display("rst=%b out=%b", rst, out);
        rst = 1'b1;
        #400 $display("rst=%b out=%b", rst, out);
        rst = 1'b0;
        #1200 $display("rst=%b out=%b", rst, out);
        rst = 1'b1;
        #400 $display("rst=%b out=%b", rst, out);
        rst = 1'b0;
        #1200 $display("rst=%b out=%b", rst, out);
        repeat(10) @(posedge clk); // repeat 10 times
        $finish;
    end
endmodule
\end{lstlisting}
\begin{figure}[H]
    \begin{center}
        \includegraphics[width=15cm]{figures/flipflop_latch.png}
        \caption{ラッチ回路の波形}
    \end{center}
\end{figure}
図ではinAとなっているものが入力Dである。設計通り、clkの立ち上がりのときのAの値が出力として保持されていることがわかる。

\subsection{2日目}
\subsubsection{ハーフアダーの設計}
ハーフアダーは1bitの加算を桁上げまで含めて行う。
\begin{figure}[H]
    \begin{center}
        \begin{circuitikz}[american currents]
            \tikzset{halfadder/.style={draw, thick, minimum height=1.5cm, minimum width=2cm, rounded corners = 0.3cm}}
            \node[halfadder] (A) at (0,0) {};
            \draw ($(A.north west)!.5!(A.west)$) to[short,l_=A,-o] ++(-1,0)
            ($(A.south west)!.5!(A.west)$) to[short,l_=B,-o] ++(-1,0)
            ($(A.north east)!.5!(A.east)$) to[short,l=S,-o] ++(1,0)
            ($(A.south east)!.5!(A.east)$) to[short,l=C,-o] ++(1,0);
        \end{circuitikz}
        \caption{ハーフアダー}
    \end{center}
\end{figure}
入力は2つのwireであり、出力は2つのwireである。入力をA,Bとすると$S=A\oplus B$、$C=A\dot B$を出力する。前日に作成したAND回路やXOR回路も使用している。
\begin{lstlisting}[caption=ハーフアダーデザイン,language=verilog]
`timescale 1ns / 1ps
module HalfAdder(
    input wire A,
    input wire B,
    output wire S,
    output wire C
    );
    xor_gate U0 (.inA(A), .inB(B), .out(S));
    and_gate U1 (.inA(A), .inB(B), .out(C));    
endmodule
\end{lstlisting}
4通りの2入力を入力し、出力を見る。
\begin{lstlisting}[caption=ハーフアダーテストベンチ,language=verilog]
    module HalfAddertestbench;
    // parameter
    parameter CYCLE = 1000; // clock cycle
    parameter HALF_CYCLE = 500; // half cycle
    parameter DLY = 500; // delay
    // wire/reg
    reg clk;
    reg  A,B;
    wire C,S;
    // DUT module
    HalfAdder  HalfAdder0(
        .A(A),
        .B(B),
        .C(C),
        .S(S)
    );    
    // clock generator
    always begin
        clk = 1'b1;
        #(HALF_CYCLE) clk = 1'b0;
        #(HALF_CYCLE);
    end
    // test scenario
    initial begin
        // for and_gate
        A = 1'b0; B = 1'b0;
        #100 $display("A=%b B=%b C=%b S=%b", A, B, C, S);
        A = 1'b1; B = 1'b0;
        #100 $display("A=%b B=%b C=%b S=%b", A, B, C, S); 
        A = 1'b0; B = 1'b1;
        #100 $display("A=%b B=%b C=%b S=%b", A, B, C, S);  
        A = 1'b1; B = 1'b1;
        #100 $display("A=%b B=%b C=%b S=%b", A, B, C, S);
        repeat(10) @(posedge clk); // repeat 10 times
        $finish;
    end
endmodule
\end{lstlisting}
\begin{figure}[H]
    \begin{center}
        \includegraphics[width=8cm]{figures/halfadder.png}
        \caption{ハーフアダー回路の波形}
    \end{center}
\end{figure}
図から値の名前が消えてしまっているが、上から順にclk、A, B, C, Sである。実際に、$C=A\oplus B$、$S=A\dot B$を表現することができていることがわかる。

\subsubsection{フルアダーの設計}
\begin{figure}[H]
    \begin{center}
        \begin{circuitikz}[american currents]
            \tikzset{halfadder/.style={draw, thick, minimum height=1.5cm, minimum width=2cm, rounded corners = 0.3cm}}
            \node[halfadder] (A) at (0,0) {};
            \draw ($(A.north west)!.5!(A.west)$) to[short,l_=A,-o] ++(-1,0)
            ($(A.south west)!.5!(A.west)$) to[short,l_=B,-o] ++(-1,0)
            ($(A.north east)!.5!(A.east)$) to[short,l=S,-o] ++(1,0)
            ($(A.south east)!.5!(A.east)$) to[short,l=C,-o] ++(1,0);
        \end{circuitikz}
        \caption{ハーフアダー}
    \end{center}
\end{figure}
\section{参考文献}
『title』\url{<url>} 2021年n月m日アクセス
\end{document}