\documentclass{ltjsarticle}
\usepackage{luatexja}
%Packages -----------------------
\usepackage{listings}
\usepackage[euler]{textgreek}
\usepackage{enumitem}
\usepackage[margin=30mm]{geometry}
\usepackage{comment}
\usepackage{hyperref}
\usepackage{float}
\usepackage{textcomp}
\usepackage{xparse}
%Circuit ------------------------
\usepackage{amsmath,amssymb}
\usepackage{graphicx}
\usepackage{siunitx}
\usepackage{tikz}
\usepackage[siunitx, RPvoltages]{circuitikz}
%Circuit ------------------------
%Command ---–--------------------
\renewcommand{\figurename}{図}
\renewcommand{\baselinestretch}{1.1}
\renewcommand\lstlistingname{ソースコード}
\renewcommand\lstlistlistingname{ソースコード}
\lstset{
    numbers=left,
    basicstyle={\ttfamily},
    identifierstyle={\small},
    commentstyle={\smallitshape},
    keywordstyle={\small\bfseries},
    ndkeywordstyle={\small},
    stringstyle={\small\ttfamily},
    frame={tb},
    breaklines=true,
    columns=[l]{fullflexible},
    xrightmargin=0\zw,
    xleftmargin=3\zw,
    numberstyle={\scriptsize},
    stepnumber=1,
    %numbersep=1,
    lineskip=-0.5ex,
    keywordstyle=\color[HTML]{e10021},
    commentstyle=\color{gray},
    emph=CascadeObjectDetector,
    emphstyle=\color{blue}
}
%Command ------------------------
%Title   ------------------------
\title{実験レポートA3}
\author{東京大学工学部電気電子工学科 03210517\ 藤田 誠之 }
\date{May 6, 2021}
%Title   ------------------------

\begin{document}

\maketitle
\begin{comment}
\section{方法}
Xilinx Vivadoを用いてHDLシミュレーションを実行した.
\section{考察課題}
\subsection{1日目}
\subsubsection{AND回路の設計}

入力は2つのwire, 出力は1つのwireである. 2つの入力の論理積を出力する。
\begin{lstlisting}[caption=AND回路デザイン,language=verilog]
`timescale 1ns / 1ps
module and_gate (
    input wire inA,
    input wire inB,
    output wire out
);
    assign out = inA & inB;
endmodule
\end{lstlisting}
2入力の全部で4通りがある。テストベンチで100nsごとに入力を変化させた。
\begin{lstlisting}[caption=AND回路テストベンチ,language=verilog]
    module testbench;
    // parameter
    parameter CYCLE = 1000; // clock cycle
    parameter HALF_CYCLE = 500; // half cycle
    parameter DLY = 500; // delay
    // wire/reg
    reg clk;
    reg inA, inB;
    wire out_and_gate;
    // DUT module
    and_gate and_gate_0 (
        .inA(inA),
        .inB(inB),
        .out(out_and_gate)
    );
    // clock generator
    always begin
        clk = 1'b1;
        #(HALF_CYCLE) clk = 1'b0;
        #(HALF_CYCLE);
    end
    // test scenario
    initial begin
        // initialize
        inA = 1'b0; inB = 1'b0; 
        // for and_gate
        inA = 1'b0; inB = 1'b0;
        #100 $display("inA=%b inB=%b out=%b", inA, inB, out_and_gate);
        inA = 1'b1; inB = 1'b0;
        #100 $display("inA=%b inB=%b out=%b", inA, inB, out_and_gate);  
        inA = 1'b0; inB = 1'b1;
        #100 $display("inA=%b inB=%b out=%b", inA, inB, out_and_gate);  
        inA = 1'b1; inB = 1'b1;
        #100 $display("inA=%b inB=%b out=%b", inA, inB, out_and_gate);

        repeat(10) @(posedge clk); // repeat 10 times
        $finish;
    end
endmodule
\end{lstlisting}

\begin{figure}[H]
    \begin{center}
        \includegraphics[width=14cm]{figures/and.png}
        \caption{AND回路の波形}
    \end{center}
\end{figure}
inA, inBの値が(0,0),(1,0),(0,1),(1,1)と変化するにつれ出力が変化していることがわかる。outの値が1となっているのはinAとinBが1のときのみであり、論理積が表現できていることがわかる。

\subsubsection{NAND回路の設計}
入力は2つのwire, 出力は1つのwireである. 2つの入力をA,Bとすると、$\overline{A\dot B}$を出力する。

\begin{lstlisting}[caption=NANDデザイン,language=verilog]
`timescale 1ns / 1ps
module nand_gate (
    input wire inA,
    input wire inB,
    output wire out
);
    assign out = ~(inA & inB);
endmodule
\end{lstlisting}
テストベンチはAND回路のときと同じものを用いた。

\begin{figure}[H]
    \begin{center}
        \includegraphics[width=14cm]{figures/nand.png}
        \caption{NAND回路の波形}
    \end{center}
\end{figure}
inA, inBの値を(0,0),(1,0),(0,1),(1,1)と変化するにつれ出力が変化していることがわかる。outの値は(0,0),(1,0),(0,1)のときに1、(1,1)のときに0となっているが、これは$\overline{A\dot B}$をこの回路が正しく表現できていることがわかる。

\subsubsection{XOR回路の設計}
入力は2つのwire, 出力は1つのwireである。2つの入力をA,Bとすると、$A \oplus B$を出力する。
\begin{lstlisting}[caption=XORデザイン, language=verilog]
`timescale 1ns / 1ps
module xor_gate (
    input wire inA,
    input wire inB,
    output wire out
);
    assign out = inA ^ inB;
endmodule
\end{lstlisting}
テストベンチはAND回路のときと同じものを用いた。
\begin{figure}[H]
    \begin{center}
        \includegraphics[width=14cm]{figures/xor.png}
        \caption{XOR回路の波形}
    \end{center}
\end{figure}
inA, inBの値を(0,0),(1,0),(0,1),(1,1)と変化するにつれ出力が変化していることがわかる。outの値はinA,inBが(1,0),(0,1)のときに1となっており、$A \oplus B$が表現できていることがわかる。
\subsubsection{フリップフロップ回路の設計}
\tikzset{sr-ff/.style={flipflop, flipflop def={
    t1=D,t3 = {\texttt{CLK}}, t6=Q, c3=1, td=rst}},
}
以下のフリップフロップ回路は、出力を反転して入力に戻す。
\begin{figure}[H]
    \begin{center}
        \begin{circuitikz}[american currents]
            \ctikzset{american inductors}
            \draw (0,0)
            node[sr-ff](FF){} (FF.up)
            (FF.pin 1) -- ++(-0.5, 0)
            to[short] ++(0, 2)
            to[short] ++(1,0) node[not port, anchor = out, rotate = 180](not1){}
            (FF.pin 6) -- ++(0.5, 0)
            to[short] ++(0, 2)
            to (not1.in) -- ++(0.5,0);
        \end{circuitikz}
        \caption{FF回路}
    \end{center}
\end{figure}
このFF回路を作成する。入力はclkとrstの2つのwireであり、出力はoutのreg一つである。clkの立ち上がりで出力値がそれまでの出力値の反転となる。ただし、clkの立ち上がり時にrstが1であればoutの値は0のままとなる。また、rstの立ち上がり時にも出力値が0となる。なお、regを用いてoutの値を保持しておくことで、Dの入力のwireが不要となる。
\begin{lstlisting}[caption=インバータ付きFF回路デザイン,language=verilog]
`timescale 1ns / 1ps
module flipflop (
    input wire clk,
    input wire rst,
    output reg q
);
initial begin
    q = 1;
end
always @(posedge clk or posedge rst)
    if (rst) begin
        q <= 1'b0;
    end else begin
        q <= ~q;
    end
endmodule
\end{lstlisting}
clkは500msごとに値が1と0で切り替わる。適当なタイミングでrstの値を1とし、rstが想定通りに機能していることを確認する。
\begin{lstlisting}[caption=インバータ付きFF回路テストベンチ,language=verilog]
    module testbench_flipflop_not;
    // parameter
    parameter CYCLE = 1000; // clock cycle
    parameter HALF_CYCLE = 500; // half cycle
    parameter DLY = 500; // delay
    // wire/reg
    reg clk;
    reg rst;
    wire out;
    // DUT module
    flipflop_not flipflop_not0(
        .clk(clk),
        .rst(rst),
        .q(out)
    );
    // clock generator
    always begin
        clk = 1'b1;
        #(HALF_CYCLE) clk = 1'b0;
        #(HALF_CYCLE);
    end
    // test scenario
    initial begin   
        rst = 1'b0;
        #2200 $display("rst=%b out=%b", rst, out);
        rst = 1'b1;
        #400 $display("rst=%b out=%b", rst, out);  
        rst = 1'b0;
        #600 $display("rst=%b out=%b", rst, out);
        rst = 1'b1;
        #400 $display("rst=%b out=%b", rst, out);
        rst = 1'b0;
        #1200 $display("rst=%b out=%b", rst, out);
        rst = 1'b1;
        #400 $display("rst=%b out=%b", rst, out);
        rst = 1'b0;
        #1200 $display("rst=%b out=%b", rst, out);
        repeat(10) @(posedge clk); // repeat 10 times
        $finish;
    end
endmodule
\end{lstlisting}
\begin{figure}[H]
    \begin{center}
        \includegraphics[width=14cm]{figures/flipflop_not.png}
        \caption{インバータ付きフリップフロップ回路の波形}
    \end{center}
\end{figure}
波形を観察すると、設計通り、clkの立ち上がりでoutの値が反転しており、またrstの立ち上がりでoutの値が0となっていることがわかる。


\subsubsection{ラッチ回路の設計}
\tikzset{sr-ff2/.style={flipflop, flipflop def={
    t1=D,t3 = {\texttt{CLK}}, t6=Q, c3=1}},
}
以下のようなラッチ回路を作成する。
\begin{figure}[H]
    \begin{center}
        \begin{circuitikz}[american currents]
            \ctikzset{american inductors}
            \draw (0,0)
            node[sr-ff2](FF){Latch} (FF.up);
        \end{circuitikz}
        \caption{ラッチ回路}
    \end{center}
\end{figure}
入力はclkとDのwireの2つ、出力は1つのregである。クロックが1であれば出力の値がDの値となり、クロックが0になってもその直前の出力値を保持する。
\begin{lstlisting}[caption=ラッチ回路デザイン,language=verilog]
`timescale 1ns / 1ps
module latch (
    input wire clk,
    input wire D,
    output reg Q
);
    always @* begin
        if (clk==1'b1)
        Q = D;
    end
endmodule
\end{lstlisting}
clkは250msごとに切り替わる。Dを適当なタイミングで切り替え、out値の挙動を観察する。
\begin{lstlisting}[caption=*****,language=verilog]
module testbench_flipflop_not;
    // parameter
    parameter CYCLE = 1000; // clock cycle
    parameter HALF_CYCLE = 500; // half cycle
    parameter DLY = 500; // delay
    // wire/reg
    reg clk;
    reg rst;
    wire out;
    // DUT module
    flipflop_not flipflop_not0(
        .clk(clk),
        .rst(rst),
        .q(out)
    );
    // clock generator
    always begin
        clk = 1'b1;
        #(HALF_CYCLE) clk = 1'b0;
        #(HALF_CYCLE);
    end
    // test scenario
    initial begin
        rst = 1'b0;
        #2200 $display("rst=%b out=%b", rst, out);
        rst = 1'b1;
        #400 $display("rst=%b out=%b", rst, out);  
        rst = 1'b0;
        #600 $display("rst=%b out=%b", rst, out);
        rst = 1'b1;
        #400 $display("rst=%b out=%b", rst, out);
        rst = 1'b0;
        #1200 $display("rst=%b out=%b", rst, out);
        rst = 1'b1;
        #400 $display("rst=%b out=%b", rst, out);
        rst = 1'b0;
        #1200 $display("rst=%b out=%b", rst, out);
        repeat(10) @(posedge clk); // repeat 10 times
        $finish;
    end
endmodule
\end{lstlisting}
\begin{figure}[H]
    \begin{center}
        \includegraphics[width=15cm]{figures/flipflop_latch.png}
        \caption{ラッチ回路の波形}
    \end{center}
\end{figure}
図ではinAとなっているものが入力Dである。設計通り、clkの立ち上がりのときのAの値が出力として保持されていることがわかる。

\subsection{2日目}
\subsubsection{ハーフアダーの設計}
ハーフアダーは1bitの加算を桁上げまで含めて行う。
\begin{figure}[H]
    \begin{center}
        \begin{circuitikz}[american currents]
            \tikzset{halfadder/.style={draw, thick, minimum height=1.5cm, minimum width=2cm, rounded corners = 0.3cm}}
            \node[halfadder] (A) at (0,0) {};
            \draw ($(A.north west)!.5!(A.west)$) to[short,l_=A,-o] ++(-1,0)
            ($(A.south west)!.5!(A.west)$) to[short,l_=B,-o] ++(-1,0)
            ($(A.north east)!.5!(A.east)$) to[short,l=S,-o] ++(1,0)
            ($(A.south east)!.5!(A.east)$) to[short,l=C,-o] ++(1,0);
        \end{circuitikz}
        \caption{ハーフアダー}
    \end{center}
\end{figure}
入力は2つのwireであり、出力は2つのwireである。入力をA,Bとすると$S=A\oplus B$、$C=A\dot B$を出力する。前日に作成したAND回路やXOR回路も使用している。
\begin{lstlisting}[caption=ハーフアダーデザイン,language=verilog]
`timescale 1ns / 1ps
module HalfAdder(
    input wire A,
    input wire B,
    output wire S,
    output wire C
    );
    xor_gate U0 (.inA(A), .inB(B), .out(S));
    and_gate U1 (.inA(A), .inB(B), .out(C));    
endmodule
\end{lstlisting}
4通りの2入力を入力し、出力を見る。
\begin{lstlisting}[caption=ハーフアダーテストベンチ,language=verilog]
    module HalfAddertestbench;
    // parameter
    parameter CYCLE = 1000; // clock cycle
    parameter HALF_CYCLE = 500; // half cycle
    parameter DLY = 500; // delay
    // wire/reg
    reg clk;
    reg  A,B;
    wire C,S;
    // DUT module
    HalfAdder  HalfAdder0(
        .A(A),
        .B(B),
        .C(C),
        .S(S)
    );    
    // clock generator
    always begin
        clk = 1'b1;
        #(HALF_CYCLE) clk = 1'b0;
        #(HALF_CYCLE);
    end
    // test scenario
    initial begin
        // for and_gate
        A = 1'b0; B = 1'b0;
        #100 $display("A=%b B=%b C=%b S=%b", A, B, C, S);
        A = 1'b1; B = 1'b0;
        #100 $display("A=%b B=%b C=%b S=%b", A, B, C, S); 
        A = 1'b0; B = 1'b1;
        #100 $display("A=%b B=%b C=%b S=%b", A, B, C, S);  
        A = 1'b1; B = 1'b1;
        #100 $display("A=%b B=%b C=%b S=%b", A, B, C, S);
        repeat(10) @(posedge clk); // repeat 10 times
        $finish;
    end
endmodule
\end{lstlisting}
\begin{figure}[H]
    \begin{center}
        \includegraphics[width=8cm]{figures/halfadder.png}
        \caption{ハーフアダー回路の波形}
    \end{center}
\end{figure}
図から値の名前が消えてしまっているが、上から順にclk、A, B, C, Sである。実際に、$C=A\oplus B$、$S=A\dot B$を表現することができていることがわかる。
\subsubsection{フルアダーの設計}
フルアダーは、ハーフアダーに過去の繰り上げの情報を付加して計算する。
\begin{figure}[H]
    \begin{center}
        \begin{circuitikz}[american currents]
            \tikzset{fulladder/.style={draw, thick, minimum height=1.5cm, minimum width=2cm, rounded corners = 0.3cm}}
            \node[fulladder] (A) at (0,0) {};
            \draw ($(A.north west)!.5!(A.west)$) to[short,l_=A,-o] ++(-1,0)
            ($(A.west)!.5!(A.west)$) to[short,l_=B,-o] ++(-1,0)
            ($(A.south west)!.5!(A.west)$) to[short,l_=$C_{in}$,-o] ++(-1,0)
            ($(A.north east)!.5!(A.east)$) to[short,l=S,-o] ++(1,0)
            ($(A.south east)!.5!(A.east)$) to[short,l=$C_{out}$,-o] ++(1,0);
        \end{circuitikz}
        \caption{ハーフアダー}
    \end{center}
\end{figure}
入力は$A$、$B$、$C_{in}$の3つのwireであり、出力は$S$、$C_{in}$の2つのwireである。ハーフアダーを以下の様に2つつなげることにより、完結に表現することができる。
\begin{figure}[H]
    \begin{center}
        \begin{circuitikz}[american currents]
            \tikzset{fulladder/.style={draw, thick, minimum height=1.5cm, minimum width=2cm, rounded corners = 0.3cm}}
            \node[fulladder] (A) at (0,0) {};
            \node[fulladder] (B) at (4,0) {};
            \draw ($(A.north west)!.5!(A.west)$) to[short,l_=A,-o] ++(-1,0)
            ($(A.south west)!.5!(A.west)$) to[short,l_=B,-o] ++(-1,0)
            ($(A.north east)!.5!(A.east)$) to[short] ++(1,0)
            to ($(B.north west)!.5!(B.west)$) -- ++ (0,0)
            ($(B.north east)!.5!(B.east)$) to[short] ++(3,0)
            to[short, l=S, -o] ++(1,0)
            ($(A.south east)!.5!(A.east)$) to[short] ++(0.5,0)
            to[short] ++(0,-1.57)
            to[short] ++(5,0) node[or port, anchor = in 2](or1){}
            ($(B.south west)!.5!(B.west)$) to[short] ++(-0.5,0)
            to[short] ++(0, -1)
            to[short] ++(-3.5,0)
            to[short,l_=$C_{in}$,-o] ++(-1,0)
            ($(B.south east)!.5!(B.east)$) to[short] ++(0.5,0)
            to[short] ++(0,-1)
            to (or1.in 1) -- ++(0,0)
            (or1.out) to[short,l=$C_{out}$,-o] ++(1,0);
        \end{circuitikz}
        \caption{ハーフアダー}
    \end{center}
\end{figure}
\begin{lstlisting}[caption=フルアダーデザイン,language=verilog]
`timescale 1ns / 1ps
module Fulladder(
    input wire CI,
    input wire A,
    input wire B,
    output wire S,
    output wire CO
    );
    wire w1;
    wire w2;
    wire w3;
    HalfAdder HalfAdder0(
        .A(A),
        .B(B),
        .S(w1),
        .C(w2)
        );
    HalfAdder HalfAdder1(
        .A(w1),
        .B(CI),
        .S(S),
        .C(w3)
        );
    assign CO = w2|w3;
endmodule
\end{lstlisting}
$A$,$B$,$C_{in}$に関してそれぞれ0,1の8通りを表現する。
\begin{lstlisting}[caption=フルアダーテストベンチ,language=verilog]
module FullAddertestbench;
    // parameter
    parameter CYCLE = 1000; // clock cycle
    parameter HALF_CYCLE = 500; // half cycle
    parameter DLY = 500; // delay
    // wire/reg
    reg clk;
    reg  A,B,CI;
    wire CO,S;
    // DUT module
    Fulladder  Fulladder0(
        .A(A),
        .B(B),
        .CI(CI),
        .CO(CO),
        .S(S)
    );    
    // clock generator
    always begin
        clk = 1'b1;
        #(HALF_CYCLE) clk = 1'b0;
        #(HALF_CYCLE);
    end
    // test scenario
    initial begin
        // for and_gate
        A = 1'b0; B = 1'b0; CI = 1'b0;
        #100 $display("A=%b B=%b CI=%b CO=%b S=%b ", A, B, CI, CO, S);
        A = 1'b1; B = 1'b0; CI = 1'b0;
        #100 $display("A=%b B=%b CI=%b CO=%b S=%b", A, B, CI, CO, S); 
        A = 1'b0; B = 1'b1; CI = 1'b0;
        #100 $display("A=%b B=%b CI=%b CO=%b S=%b", A, B, CI, CO, S);  
        A = 1'b1; B = 1'b1; CI = 1'b0;
        #100 $display("A=%b B=%b CI=%b CO=%b S=%b", A, B, CI, CO, S);
        A = 1'b0; B = 1'b0; CI = 1'b1;
        #100 $display("A=%b B=%b CI=%b CO=%b S=%b ", A, B, CI, CO, S);
        A = 1'b1; B = 1'b0; CI = 1'b1;
        #100 $display("A=%b B=%b CI=%b CO=%b S=%b", A, B, CI, CO, S); 
        A = 1'b0; B = 1'b1; CI = 1'b1;
        #100 $display("A=%b B=%b CI=%b CO=%b S=%b", A, B, CI, CO, S);  
        A = 1'b1; B = 1'b1; CI = 1'b1;
        #100 $display("A=%b B=%b CI=%b CO=%b S=%b", A, B, CI, CO, S);
        repeat(10) @(posedge clk); // repeat 10 times
        $finish;
    end
endmodule
\end{lstlisting}
\begin{figure}[H]
    \begin{center}
        \includegraphics[width=10cm]{figures/fulladder.png}
        \caption{フルアダー回路の波形}
    \end{center}
\end{figure}
上から$A$、$B$、$C_{in}$、$S$、$C_{out}$である。入力に合致した値が得られていることがわかる。
\subsubsection{4-bit リップルキャリアアダーの設計}
フルアダーを4つ並べて繋げたものである。

\begin{figure}[H]
    \begin{center}
        \begin{circuitikz}[american currents]
            \tikzset{one bit adder/.style={muxdemux, muxdemux def={Lh=4, NL=2, Rh=2, NR=1, NB=1, NT=1, w=1.5, inset w=0.5, inset Lh=2, inset Rh=0, square pins=1}}}
            \draw
            node[one bit adder, rotate=-90](D) at (0,0) {A}
            node[one bit adder, rotate=-90](C) at (3,0) {B}
            node[one bit adder, rotate=-90](B) at (6,0) {C}
            node[one bit adder, rotate=-90](A) at (9,0) {D}
            (A.tpin 1) to(B.bpin 1) -- ++(0,0)
            (B.tpin 1) to(C.bpin 1) -- ++(0,0)
            (C.tpin 1) to(D.bpin 1) -- ++(0,0);
        \end{circuitikz}
        \caption{4-bit リップルキャリアアダー}
    \end{center}
\end{figure}
Aの初期値は0とし、4bitの16進数として表示された値をそれぞれの桁の値をそれぞれ対応するフルアダーに入力する。Dの$C_{out}$は最終的な計算結果のキャリーとなることがわかる。
\begin{lstlisting}[caption=4-bit リップルキャリアアダーデザイン,language=verilog]
`timescale 1ns / 1ps
module Addripple4(
    input wire [3:0]a,
    input wire [3:0]b,
    output wire [3:0]s,
    output wire of
    );
    Fulladder Fulladder0(.S(s[0]),.B(b[0]),.A(a[0]),.CI(0),.CO(w0));
    Fulladder Fulladder1(.S(s[1]),.B(b[1]),.A(a[1]),.CI(w0),.CO(w1));
    Fulladder Fulladder2(.S(s[2]),.B(b[2]),.A(a[2]),.CI(w1),.CO(w2));
    Fulladder Fulladder3(.S(s[3]),.B(b[3]),.A(a[3]),.CI(w2),.CO(of));
endmodule
\end{lstlisting}
AとBが4bitずつの$2^8$通りを計算するテストベンチを作成した。
\begin{lstlisting}[caption=*****,language=verilog]
    module testbench_addripple4;
    // parameter
    parameter CYCLE = 1000; // clock cycle
    parameter HALF_CYCLE = 500; // half cycle
    parameter DLY = 500; // delay
    // wire/reg
    reg clk;
    reg  [3:0]A,B;
    wire C;
    wire [3:0]S;
    // DUT module
    Addripple4 Addripple4_0(A,B,S,C);
    // clock generator
    always begin
        clk = 1'b1;
        #(HALF_CYCLE) clk = 1'b0;
        #(HALF_CYCLE);
    end
     integer i;
     integer j;
    // test scenario
    initial begin
        // initialize
        A = 4'b0; B = 4'b0;
        // for add4_gate
        for(i = 0; i<16; i = i+1) begin
          for(j = 0; j<16; j = j+1)begin
            A = i; B = j;
            #40 $display("A=%b B=%b C=%b S=%b", A, B, C, S);
          end        
        end
        repeat(10) @(posedge clk); // repeat 10 times
        $finish;
    end
endmodule
\end{lstlisting}
\begin{figure}[H]
    \begin{center}
        \includegraphics[width=14cm]{figures/addripple4.png}
        \caption{4-bit リップルキャリアアダー}
    \end{center}
\end{figure}
256通りすべてを網羅した。入力と出力A,B,Sは16進法で表されており、また計算全体としてキャリーが発生する場合はCがと1となっており、これがオーバーフローの検出として機能していることがわかる。
\subsubsection{4-bit アダーの設計}
フルアダーの組み合わせとしてではなく、1つの回路として4-bitの加算器を作成する。
\begin{lstlisting}[caption=4-bit アダー,language=verilog]
`timescale 1ns / 1ps
module Add4(a,b,s,carry); 
    input [3:0]a,b;
    output [3:0]s;
    output carry;
    assign {carry, s} = a + b;
endmodule    
\end{lstlisting}
4-bitリップルキャリアアダーに比べ、とても簡潔に表現できている。テストベンチは4-bit リップルキャリアアダーと同じものを用いた。
\begin{figure}[H]
    \begin{center}
        \includegraphics[width=14cm]{figures/addripple4.png}
        \caption{フルアダー回路の波形}
    \end{center}
\end{figure}
結果、同じテストベンチ4-bitリップルキャリアアダーと同一の結果が得られた。
\end{comment}
\subsection{3日目}
\subsubsection{動作の違いの調査}
以下の回路は入力が2つのwire、出力が2つのwireの回路である。
\begin{lstlisting}[caption=*****,language=verilog]
`timescale 1ns / 1ps
module example (
    input wire inA,
    input wire clk,
    output wire out1,
    output wire out2
);
    reg out1_reg, out2_reg;
    always @(posedge clk) begin
        out1_reg <= inA;
    end
    always @(*) begin
        out2_reg <= inA;
    end
    assign out1 = out1_reg;
    assign out2 = out2_reg;
endmodule
\end{lstlisting}
ソースコードを調べると、posedge内でout1regを決めていることがわかるため、フリップフロップはout1であると予想できる。これを確かめるテストベンチを作成した。40msごとに入力を切り替えている。
\begin{lstlisting}[caption=*****,language=verilog]
module testbench_example;
  wire	o1,o2;
  reg clk;
  reg	a;
  always begin
        clk = 1'b1;
        #50 clk = 1'b0;
        #50;
  end
  integer i = 0;
  initial begin
	a <= 0;
	for (i = 0; i < 40; i = i+1) begin
	   a <= i%2;
	   #28;
	end
	repeat(2) @(posedge clk);
	$finish;
  end
  example example0( a,clk,o1,o2 );
endmodule
\end{lstlisting}
\begin{figure}[H]
    \begin{center}
        \includegraphics[width=15cm]{figures/example.png}
        \caption{example回路の波形}
    \end{center}
\end{figure}
実際、out1がフリップフロップ回路であることが確認できた。
\subsubsection{基本的な順序回路の設計}
\begin{figure}[H]
    \begin{center}
        \includegraphics[width=10cm]{figures/junjo.png}
        \caption{順序回路}
    \end{center}
\end{figure}

\section{参考文献}
『title』\url{<url>} 2021年n月m日アクセス
\end{document}