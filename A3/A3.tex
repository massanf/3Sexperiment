\documentclass{ltjsarticle}
\usepackage{luatexja}
%Packages -----------------------
\usepackage{listings}
\usepackage[euler]{textgreek}
\usepackage{enumitem}
\usepackage[margin=30mm]{geometry}
\usepackage{comment}
\usepackage{hyperref}
\usepackage{float}
\usepackage{textcomp}
\usepackage{xparse}
%Circuit ------------------------
\usepackage{amsmath,amssymb}
\usepackage{graphicx}
\usepackage{siunitx}
\usepackage{tikz}
\usepackage[siunitx, RPvoltages]{circuitikz}
%Circuit ------------------------
%Command ---–--------------------
\renewcommand{\figurename}{図}
\renewcommand{\baselinestretch}{1.1}
\renewcommand\lstlistingname{ソースコード}
\renewcommand\lstlistlistingname{ソースコード}
\lstset{
    numbers=left,
    basicstyle={\ttfamily},
    identifierstyle={\small},
    commentstyle={\smallitshape},
    keywordstyle={\small\bfseries},
    ndkeywordstyle={\small},
    stringstyle={\small\ttfamily},
    frame={tb},
    breaklines=true,
    columns=[l]{fullflexible},
    xrightmargin=0\zw,
    xleftmargin=3\zw,
    numberstyle={\scriptsize},
    stepnumber=1,
    %numbersep=1,
    lineskip=-0.5ex,
    keywordstyle=\color[HTML]{e10021},
    commentstyle=\color{gray},
    emph=CascadeObjectDetector,
    emphstyle=\color{blue}
}
%Command ------------------------
%Title   ------------------------
\title{実験レポートA3}
\author{東京大学工学部電気電子工学科 03210517\ 藤田 誠之 }
\date{May 6, 2021}
%Title   ------------------------

\begin{document}

\maketitle
\section{方法}
Xilinx Vivadoを用いてHDLシミュレーションを実行した.
\section{考察課題}
\subsection{1日目}
\subsubsection{AND回路の設計}

入力は2つのwire, 出力は1つのwireである. 2つの入力の論理積を出力する。
\begin{lstlisting}[caption=AND回路デザイン,language=verilog]
`timescale 1ns / 1ps
module and_gate (
    input wire inA,
    input wire inB,
    output wire out
);
    assign out = inA & inB;
endmodule
\end{lstlisting}
2入力の全部で4通りがある。テストベンチで100nsごとに入力を変化させた。
\begin{lstlisting}[caption=AND回路テストベンチ,language=verilog]
    module testbench;
    // parameter
    parameter CYCLE = 1000; // clock cycle
    parameter HALF_CYCLE = 500; // half cycle
    parameter DLY = 500; // delay
    // wire/reg
    reg clk;
    reg inA, inB;
    wire out_and_gate;
    // DUT module
    and_gate and_gate_0 (
        .inA(inA),
        .inB(inB),
        .out(out_and_gate)
    );
    // clock generator
    always begin
        clk = 1'b1;
        #(HALF_CYCLE) clk = 1'b0;
        #(HALF_CYCLE);
    end
    // test scenario
    initial begin
        // initialize
        inA = 1'b0; inB = 1'b0; 
        // for and_gate
        inA = 1'b0; inB = 1'b0;
        #100 $display("inA=%b inB=%b out=%b", inA, inB, out_and_gate);
        inA = 1'b1; inB = 1'b0;
        #100 $display("inA=%b inB=%b out=%b", inA, inB, out_and_gate);  
        inA = 1'b0; inB = 1'b1;
        #100 $display("inA=%b inB=%b out=%b", inA, inB, out_and_gate);  
        inA = 1'b1; inB = 1'b1;
        #100 $display("inA=%b inB=%b out=%b", inA, inB, out_and_gate);

        repeat(10) @(posedge clk); // repeat 10 times
        $finish;
    end
endmodule
\end{lstlisting}

\begin{figure}[H]
    \begin{center}
        \includegraphics[width=14cm]{figures/and.png}
        \caption{AND回路の波形}
    \end{center}
\end{figure}
inA, inBの値が(0,0),(1,0),(0,1),(1,1)と変化するにつれ出力が変化していることがわかる。outの値が1となっているのはinAとinBが1のときのみであり、論理積が表現できていることがわかる。

\subsubsection{NAND回路の設計}
入力は2つのwire, 出力は1つのwireである. 2つの入力をA,Bとすると、$\overline{A\dot B}$を出力する。

\begin{lstlisting}[caption=NANDデザイン,language=verilog]
`timescale 1ns / 1ps
module nand_gate (
    input wire inA,
    input wire inB,
    output wire out
);
    assign out = ~(inA & inB);
endmodule
\end{lstlisting}
テストベンチはAND回路のときと同じものを用いた。

\begin{figure}[H]
    \begin{center}
        \includegraphics[width=14cm]{figures/nand.png}
        \caption{NAND回路の波形}
    \end{center}
\end{figure}
inA, inBの値を(0,0),(1,0),(0,1),(1,1)と変化するにつれ出力が変化していることがわかる。outの値は(0,0),(1,0),(0,1)のときに1、(1,1)のときに0となっているが、これは$\overline{A\dot B}$をこの回路が正しく表現できていることがわかる。

\subsubsection{XOR回路の設計}
入力は2つのwire, 出力は1つのwireである。2つの入力をA,Bとすると、$A \oplus B$を出力する。
\begin{lstlisting}[caption=XORデザイン, language=verilog]
`timescale 1ns / 1ps
module xor_gate (
    input wire inA,
    input wire inB,
    output wire out
);
    assign out = inA ^ inB;
endmodule
\end{lstlisting}
テストベンチはAND回路のときと同じものを用いた。
\begin{figure}[H]
    \begin{center}
        \includegraphics[width=14cm]{figures/xor.png}
        \caption{XOR回路の波形}
    \end{center}
\end{figure}
inA, inBの値を(0,0),(1,0),(0,1),(1,1)と変化するにつれ出力が変化していることがわかる。outの値はinA,inBが(1,0),(0,1)のときに1となっており、$A \oplus B$が表現できていることがわかる。
\subsection{2日目}
\subsubsection{フリップフロップ回路の設計}
\tikzset{sr-ff/.style={flipflop, flipflop def={
    t1=D,t3 = {\texttt{CLK}}, t6=Q, c3=1, td=rst}},
}
以下のフリップフロップ回路は、出力を反転して入力に戻す。
\begin{figure}[H]
    \begin{center}
        \begin{circuitikz}[american currents]
            \ctikzset{american inductors}
            \draw (0,0)
            node[sr-ff](FF){} (FF.up)
            (FF.pin 1) -- ++(-0.5, 0)
            to[short] ++(0, 2)
            to[short] ++(1,0) node[not port, anchor = out, rotate = 180](not1){}
            (FF.pin 6) -- ++(0.5, 0)
            to[short] ++(0, 2)
            to (not1.in) -- ++(0.5,0);
        \end{circuitikz}
        \caption{FF回路}
    \end{center}
\end{figure}
このFF回路を作成する。
\begin{lstlisting}[caption=*****,language=verilog]
\end{lstlisting}

\section{参考文献}
『title』\url{<url>} 2021年n月m日アクセス
\end{document}