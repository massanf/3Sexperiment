\documentclass[a4j,dvipdfmx]{article}
%Format   -----------------------
\usepackage[utf8]{inputenc}
\usepackage[euler]{textgreek}
\usepackage{enumitem}
\usepackage[margin=30mm]{geometry}
\usepackage{comment}
\usepackage[dvipdfmx]{hyperref}
\usepackage{float}
\usepackage{subcaption}
\usepackage{textcomp}
\usepackage{palatino}
\usepackage{xparse}
%Format   -----------------------

\def\replace#1#2#3{%
 \def\tmp##1#2{##1#3\tmp}%
   \tmp#1\stopreplace#2\stopreplace}
\def\stopreplace#1\stopreplace{}

%Circuit ------------------------
%\begin{comment}
\usepackage{amsmath,amssymb}
\usepackage[dvipdfmx]{graphicx}
\usepackage{siunitx}
\usepackage{tikz}
\usepackage{circuitikz}
%\end{comment}
%Circuit ------------------------

%Command ---–--------------------
\renewcommand{\figurename}{図}
%--------------------------------

%Title   ------------------------
\title{実験レポートA1}
\author{東京大学工学部電気電子工学科 03210517\\ 藤田 誠之 }
\date{April 15, 2021}
%Title   ------------------------
\renewcommand{\baselinestretch}{1.1}
\begin{document}

\tikzset{component/.style={draw,thick,circle,fill=white,minimum size =0.75cm,inner sep=0pt}}

\begin{figure}[H]
  \begin{center}
    \begin{circuitikz}[american currents]
     \ctikzset{american inductors}
	  \draw (0,4)
      to[battery1] (0,0)
      to[short] (5,0)
	  to[short] (5,2) 
	  node[component]{V} 
	  to[short] (5,4)
	  to[short] (2.5,4) node[component]{A} to [short] (0,4);
	  \draw(5,4)
	  to[short] (6,4);
    \draw(7,2)
	  node[njfet](njfet){}
	  (njfet.base) node[anchor=east] {G}
	  (njfet.collector) node[anchor=south] {D}
	  (njfet.emitter) node[anchor=west] {S};
	  to[short] (3,0);
	  \draw(6,4)
	  to[short] (njfet.base);
    \draw(4,0)
    to[short] (7,0)
    to[short] (njfet.emitter);
	  \draw ($(njfet)-(0.18,0)$) circle [radius=18pt];
    \end{circuitikz}
    \caption{a順方向}
  \end{center}
\end{figure}

\begin{figure}[H]
  \begin{center}
    \begin{circuitikz}[american currents]
     \ctikzset{american inductors}
    \draw (0,4)
      to[battery1] (0,0)
      to[short] (5,0)
    to[short] (5,2) 
    node[component]{V} 
    to[short] (5,4)
    to[short] (2.5,4) node[component]{A} to [short] (0,4);
    \draw(5,4)
    to[short] (6,4);
    \draw(7,2)
    node[njfet](njfet){}
    (njfet.base) node[anchor=east] {G}
    (njfet.collector) node[anchor=south] {D}
    (njfet.emitter) node[anchor=east] {S};
    to[short] (3,0);
    \draw(6,4)
    to[short] (8,4)
    to[short] (8,1.25)
    to[short] (njfet.emitter);
    \draw(4,0)
    to[short] (6,0)
    to[short] (njfet.base);
    \draw ($(njfet)-(0.18,0)$) circle [radius=18pt];
    \end{circuitikz}
    \caption{a逆方向}
  \end{center}
\end{figure}

\begin{figure}[H]
  \begin{center}
    \begin{circuitikz}[american currents]
     \ctikzset{american inductors}
    \draw (0,4)
      to[battery1] (0,0)
      to[short] (1,0)
    to[short] (1,2) 
    node[component]{V} 
    to[short] (1,4);
    \draw (0,4)
    to[short] (3,4) node[component]{A} to [short] (6,4);
    \draw(5,4)
    to[short] (6,4);
    \draw(7,2)
    node[njfet](njfet){}
    (njfet.base) node[anchor=east] {G}
    (njfet.collector) node[anchor=south] {D}
    (njfet.emitter) node[anchor=west] {S};
    to[short] (3,0);
    \draw(6,4)
    to[short] (njfet.base);
    \draw(1,0)
    to[short] (7,0)
    to[short] (njfet.emitter);
    \draw ($(njfet)-(0.18,0)$) circle [radius=18pt];
    \end{circuitikz}
    \caption{b順方向}
  \end{center}
\end{figure}

\begin{figure}[H]
  \begin{center}
    \begin{circuitikz}[american currents]
     \ctikzset{american inductors}
    \draw (0,4)
      to[battery1] (0,0)
      to[short] (1,0)
    to[short] (1,2) 
    node[component]{V} 
    to[short] (1,4);
    \draw (0,4)
    to[short] (3,4) node[component]{A} to [short] (6,4);
    \draw(5,4)
    to[short] (6,4);
    \draw(7,2) node[njfet](njfet){}
    (njfet.base) node[anchor=east] {G}
    (njfet.collector) node[anchor=south] {D}
    (njfet.emitter) node[anchor=east] {S};
    to[short] (3,0);
    \draw(6,4)
    to[short] (8,4)
    to[short] (8,1.25)
    to[short] (njfet.emitter);
    \draw(1,0)
    to[short] (6,0)
    to[short] (njfet.base);
    \draw ($(njfet)-(0.18,0)$) circle [radius=18pt];
    \end{circuitikz}
    \caption{b逆方向}
  \end{center}
\end{figure}


\begin{figure}[H]
  \begin{center}
    \begin{circuitikz}[american currents]
     \ctikzset{american inductors}
    \draw (0,0)
    to[battery1] (0,2)
    to[short] (1,2)
    to[short] (1,1) node[component](v2){V}
    to[short] (1,0)
    to[short] (0,0);
    \draw ($(v2)+(0.4,0)$) node[anchor=west] {$V_{GS}$};
    \draw(4,2.27) node[njfet](njfet){}
    (njfet.base) node[anchor=north] {G}
    (njfet.collector) node[anchor=east] {D}
    (njfet.emitter) node[anchor=east] {S};
    \draw ($(njfet)-(0.18,0)$) circle [radius=18pt];
    \draw (0,2)
    to[short] (njfet.base);
    \draw (1,0)
    to[short] (4,0)
    to[short] (njfet.emitter);
    \draw (njfet.collector)
    to[short] (4,4)
    to[short] (5,4)
    to[short] (5,2)
    node[component](v1){V}
    to[short] (5,0)
    to[short] (4,0);
    \draw ($(v1)+(0.4,0)$) node[anchor=west] {$V_{DS}$};
    \draw (5,4)
    to[short] (6,4)
    node[component]{A}
    to[short] (7,4)
    to[battery1] (7,0)
    to[short] (4,0);
    \end{circuitikz}
    \caption{2}
  \end{center}
\end{figure}

\maketitle

\section{参考文献}
『title』\url{<url>} 2021年n月m日アクセス
\end{document}


\begin{comment}
--------------------------------
--見出し
\section{実験方法}
--番号を振る
\begin{enumerate}[label={(\arabic*)}]
\item
--画像の挿入
\begin{figure}[H]
    \begin{center}
     	\includegraphics[width=8cm]{figures/name}
        \caption{title)}
    \end{center}
\end{figure}

--回路(例)
\begin{figure}[H]
  \begin{center}
    \begin{circuitikz}[american currents]
     \ctikzset{american inductors}
	  \draw (0,0)
      to[sV=$V_s$] (0,2)
      to[L=$L_2$] (6,2)
      to[C=$$
      to[european resistor=$1\Omega$] (6,0)
      to[short] (3,0);
    \end{circuitikz}
    \caption{3次規格化0-R型LPF}
  \end{center}
\end{figure}

--表
\begin{table}[H]
    \begin{center}
        \begin{tabular}{|l|l|c||} \hline
            11 & 12 & 13 \\ \hline
            21 & 22 & 23 \\ \hline
            31 & 32 & 33\\ \hline
        \end{tabular}
        \caption{title}
    \end{center}
\end{table}
\end{comment}

























