\documentclass[a4j,dvipdfmx]{article}
%Format   -----------------------
\usepackage[utf8]{inputenc}
\usepackage[euler]{textgreek}
\usepackage{enumitem}
\usepackage[margin=30mm]{geometry}
\usepackage{comment}
\usepackage[dvipdfmx]{hyperref}
\usepackage{float}
\usepackage{subcaption}
\usepackage{textcomp}
\usepackage{palatino}
\usepackage{xparse}
%Format   -----------------------

%Circuit ------------------------
%\begin{comment}
\usepackage{amsmath,amssymb}
\usepackage[dvipdfmx]{graphicx}
\usepackage{siunitx}
\usepackage{tikz}
\usepackage{circuitikz}
%\end{comment}
%Circuit ------------------------

%Command ---–--------------------
\renewcommand{\figurename}{図}
%--------------------------------

%Title   ------------------------
\title{実験レポートA1}
\author{東京大学工学部電気電子工学科 03210517\ 藤田 誠之 }
\date{May 15, 2021}
%Title   ------------------------
\renewcommand{\baselinestretch}{1.1}
\begin{document}

\maketitle

\section{}

\section{参考文献}
『title』\url{<url>} 2021年n月m日アクセス
\end{document}


\begin{comment}
--------------------------------
--見出し
\section{実験方法}
--番号を振る
\begin{enumerate}[label={(\arabic*)}]
\item
--画像の挿入
\begin{figure}[H]
\begin{center}
\includegraphics[width=8cm]{figures/name}
\caption{title)}
\end{center}
\end{figure}

--回路(例)
\begin{figure}[H]
\begin{center}
\begin{circuitikz}[american currents]
\ctikzset{american inductors}
\draw (0,0)
to[sV=V_s$] (0,2)
to[L=L_2$] (6,2)
to[C=$$
to[european resistor=\Omega$] (6,0)
to[short] (3,0);
\end{circuitikz}
\caption{3次規格化0-R型LPF}
\end{center}
\end{figure}

--表
\begin{table}[H]
\begin{center}
\begin{tabular}{|l|l|c||} \hline
11 & 12 & 13 \ \hline
21 & 22 & 23 \ \hline
31 & 32 & 33\ \hline
\end{tabular}
\caption{title}
\end{center}
\end{table}
\end{comment}