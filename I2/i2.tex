\documentclass{ltjsarticle}
\usepackage{luatexja}
%Packages -----------------------
\usepackage{listings}
\usepackage[euler]{textgreek}
\usepackage{enumitem}
\usepackage[margin=30mm]{geometry}
\usepackage{comment}
\usepackage{hyperref}
\usepackage{float}
\usepackage{textcomp}
\usepackage{xparse}
%Circuit ------------------------
\usepackage{amsmath,amssymb}
\usepackage{graphicx}
\usepackage{siunitx}
\usepackage{tikz}
\usepackage[siunitx, RPvoltages]{circuitikz}
%Circuit ------------------------
%Command ---–--------------------
\renewcommand{\figurename}{図}
\renewcommand{\baselinestretch}{1.1}
\lstset{
    numbers=left,
    basicstyle={\ttfamily},
    identifierstyle={\small},
    commentstyle={\smallitshape},
    keywordstyle={\small\bfseries},
    ndkeywordstyle={\small},
    stringstyle={\small\ttfamily},
    frame={tb},
    breaklines=true,
    columns=[l]{fullflexible},
    xrightmargin=0\zw,
    xleftmargin=3\zw,
    numberstyle={\scriptsize},
    stepnumber=1,
    %numbersep=1,
    lineskip=-0.5ex,
    keywordstyle=\color[HTML]{e10021},
    commentstyle=\color{gray},
    emph=CascadeObjectDetector,
    emphstyle=\color{blue}
}
%Command ------------------------
%Title   ------------------------
\begin{titlepage}
    \title{実験レポートI2}
    \author{東京大学工学部電気電子工学科 03210517\\藤田 誠之\\~\\ 共同実験者  河野亮介・梶谷優貴・パリナヨックサタユ}
    \date{June 13, 2021}
  \end{titlepage}
%Title   ------------------------

\begin{document}

\maketitle

\subsection*{本課題5.9}

サーバー側、クライアント側のPCでそれぞれ以下のプログラムを実行した。
\begin{lstlisting}[caption=サーバー側,language=bash]
$ nc -l 9999 > test_output.txt
\end{lstlisting}
\begin{lstlisting}[caption=クライアント側,language=bash]
$ nc 192.168.0.6 9999 < test_input.txt
\end{lstlisting}
その結果、サーバー側のパソコンのtest\_output.txtにクライアント側のtest\_input.fileの内容をコピーすることができた。


\subsection*{本課題5.10}
サーバー側、クライアント側のPCでそれぞれ以下のプログラムを実行した。
\begin{lstlisting}[caption=サーバー側,language=bash]
$ rec -t raw -b 16 -c 1 -e s -r 44100 - | nc -l 9999
\end{lstlisting}
\begin{lstlisting}[caption=クライアント側,language=bash]
$ nc 192.168.0.6 9999 | play -t raw -b 16 -c 1 -e s -r 44100 -
\end{lstlisting}
サーバー側ではrecで収録したデータをncに送り、クライアント側ではncから受け取ったデータをplayで再生している。その結果、サーバー側の音声をクライアント側から聴くことができた。




\section{参考文献}
『title』\url{<url>} 2021年n月m日アクセス
\end{document}